% Options for packages loaded elsewhere
\PassOptionsToPackage{unicode}{hyperref}
\PassOptionsToPackage{hyphens}{url}
\PassOptionsToPackage{dvipsnames,svgnames*,x11names*}{xcolor}
%
\documentclass[
  12pt,
  spanish,
]{article}
\usepackage{lmodern}
\usepackage{amssymb,amsmath}
\usepackage{ifxetex,ifluatex}
\ifnum 0\ifxetex 1\fi\ifluatex 1\fi=0 % if pdftex
  \usepackage[T1]{fontenc}
  \usepackage[utf8]{inputenc}
  \usepackage{textcomp} % provide euro and other symbols
\else % if luatex or xetex
  \usepackage{unicode-math}
  \defaultfontfeatures{Scale=MatchLowercase}
  \defaultfontfeatures[\rmfamily]{Ligatures=TeX,Scale=1}
  \setmainfont[]{Ubuntu}
  \setmonofont[]{Ubuntu Mono}
\fi
% Use upquote if available, for straight quotes in verbatim environments
\IfFileExists{upquote.sty}{\usepackage{upquote}}{}
\IfFileExists{microtype.sty}{% use microtype if available
  \usepackage[]{microtype}
  \UseMicrotypeSet[protrusion]{basicmath} % disable protrusion for tt fonts
}{}
\makeatletter
\@ifundefined{KOMAClassName}{% if non-KOMA class
  \IfFileExists{parskip.sty}{%
    \usepackage{parskip}
  }{% else
    \setlength{\parindent}{0pt}
    \setlength{\parskip}{6pt plus 2pt minus 1pt}}
}{% if KOMA class
  \KOMAoptions{parskip=half}}
\makeatother
\usepackage{xcolor}
\IfFileExists{xurl.sty}{\usepackage{xurl}}{} % add URL line breaks if available
\IfFileExists{bookmark.sty}{\usepackage{bookmark}}{\usepackage{hyperref}}
\hypersetup{
  pdftitle={Apuntes de web2py},
  pdfauthor={Sergio Alvariño salvari@gmail.com},
  pdflang={es-ES},
  colorlinks=true,
  linkcolor=Maroon,
  filecolor=Maroon,
  citecolor=Blue,
  urlcolor=Blue,
  pdfcreator={LaTeX via pandoc}}
\urlstyle{same} % disable monospaced font for URLs
\usepackage[a4paper]{geometry}
\usepackage{graphicx,grffile}
\makeatletter
\def\maxwidth{\ifdim\Gin@nat@width>\linewidth\linewidth\else\Gin@nat@width\fi}
\def\maxheight{\ifdim\Gin@nat@height>\textheight\textheight\else\Gin@nat@height\fi}
\makeatother
% Scale images if necessary, so that they will not overflow the page
% margins by default, and it is still possible to overwrite the defaults
% using explicit options in \includegraphics[width, height, ...]{}
\setkeys{Gin}{width=\maxwidth,height=\maxheight,keepaspectratio}
% Set default figure placement to htbp
\makeatletter
\def\fps@figure{htbp}
\makeatother
\setlength{\emergencystretch}{3em} % prevent overfull lines
\providecommand{\tightlist}{%
  \setlength{\itemsep}{0pt}\setlength{\parskip}{0pt}}
\setcounter{secnumdepth}{5}
\ifxetex
  % Load polyglossia as late as possible: uses bidi with RTL langages (e.g. Hebrew, Arabic)
  \usepackage{polyglossia}
  \setmainlanguage[]{spanish}
\else
  \usepackage[shorthands=off,main=spanish]{babel}
\fi

\title{Apuntes de web2py}
\author{Sergio Alvariño
\href{mailto:salvari@gmail.com}{\nolinkurl{salvari@gmail.com}}}
\date{Agosto-2019}

\begin{document}
\maketitle
\begin{abstract}
Apuntes de web2py: Un framework para desarrollo de aplicaciones web
\end{abstract}

{
\hypersetup{linkcolor=}
\setcounter{tocdepth}{3}
\tableofcontents
}
\hypertarget{introducciuxf3n}{%
\section{Introducción}\label{introducciuxf3n}}

\href{http://www.web2py.com/}{web2py} es un \emph{framework} para
facilitar el desarrollo de aplicaciones web escrito en Python.

\textbf{\emph{web2py}} funciona correctamente en Python 3. Su curva de
aprendizaje no es tan empinada como la de
\href{https://www.djangoproject.com/}{Django} (que es el
\emph{framework} de aplicaciones web de referencia en Python) y en
muchos sentidos es más moderno que Django.

\textbf{\emph{web2py}} tiene una
\href{http://www.web2py.com/init/default/documentation}{documentación}
muy completa y actualizada (disponible también en castellano) y sobre
todo una comunidad de usuarios y desarrolladores muy activa y que
responden con rapidez a las dudas que puedas plantear.

\textbf{web2py} está basado en el modelo
\href{https://es.wikipedia.org/wiki/Modelo\%E2\%80\%93vista\%E2\%80\%93controlador}{MVC}

\textbf{web2py} incorpora \emph{Bootstrap 4}

\hypertarget{referencias}{%
\subsection{Referencias}\label{referencias}}

\begin{itemize}
\tightlist
\item
  \href{http://www.web2py.com/init/default/documentation}{Página de
  documentación y recursos de web2py, con enlaces a grupos de usuarios y
  tutoriales}
\item
  \href{https://martinfowler.com/eaaDev/uiArchs.html}{Evolución del
  modelo MVC}
\item
  \href{https://nomadphp.com/blog/60/working-with-the-thin-controller-and-fat-model-concept-in-laravel}{Fat
  models and thin controllers}
\item
  \href{https://nomadphp.com/blog/60/working-with-the-thin-controller-and-fat-model-concept-in-laravel}{Crítica
  del mantra}
\end{itemize}

\hypertarget{empezar-ruxe1pido}{%
\section{Empezar rápido}\label{empezar-ruxe1pido}}

\hypertarget{instalaciuxf3n}{%
\subsection{Instalación}\label{instalaciuxf3n}}

Vamos a ver el proceso de instalación de una instancia de
\textbf{\emph{web2py}} en modo \emph{standalone}. \textbf{\emph{web2py}}
instalado de esta forma es ideal para entornos de desarrollo. Para un
entorno de producción puede ser más conveniente instalar
\textbf{\emph{web2py}} tras un servidor web como
\href{https://www.apache.org/}{Apache} o
\href{https://www.nginx.com/}{Nginx}, pero dependiendo de la carga de
trabajo y de como administres tus sistemas puede ser mejor opción usarlo
\emph{standalone} también en producción.

\begin{enumerate}
\def\labelenumi{\arabic{enumi}.}
\item
  Creamos un entorno virtual

  Como ya hemos comentado \textbf{\emph{web2py}} funciona ya en Python
  3. Y en cualquier caso, con Python nunca está de mas encapsular
  nuestras pruebas y desarrollos en un entorno virtual.\footnote{Los
    siguientes comandos asumen que tienes instalado
    \emph{virtualenvwrapper} como recomendamos en la guía de
    postinstalación de Linux Mint, si no lo tienes te recomendamos crear
    un virtualenv con los comandos tradicionales} Así que creamos el
  virtualenv que llamaremos \emph{web2py}:

\begin{verbatim}
mkvirtualenv -p `which python3` web2py
\end{verbatim}
\item
  Bajamos el programa de la web de Web2py y descomprimimos el framework:

\begin{verbatim}
# creamos un directorio (cambia el path a tu gusto)
mkdir web2py_test
cd web2py_test

# bajamos el programa de la web y descomprimimos
wget https://mdipierro.pythonanywhere.com/examples/static/web2py_src.zip

# opcionalmente borramos el zip, aunque sería mejor guardarlo
# por si queremos hacer nuevas instalaciones
rm web2py_src.zip
\end{verbatim}
\item
  Generamos certificados para el protocolo \emph{ssl}:

  Para usar con comodidad web2py conviene que nos generemos unos
  certificados para gestionar el ssl:

\begin{verbatim}
# nos movemos al directorio de web2py
cd web2py

openssl genrsa -out server.key 2048
openssl req -new -key server.key -out server.csr

Country Name (2 letter code) [AU]:ES
State or Province Name (full name) [Some-State]:A Coruna
Locality Name (eg, city) []:A Coruna
Organization Name (eg, company) [Internet Widgits Pty Ltd]:BricoLabs
Organizational Unit Name (eg, section) []:Division de Hackeo
Common Name (e.g. server FQDN or YOUR name) []:testServer@bricolabs.cc
Email Address []:contacto@bricolabs.cc

Please enter the following 'extra' attributes
to be sent with your certificate request
A challenge password []:secret1t05
An optional company name[]:Asociacion BricoLabs
\end{verbatim}

  Y ahora ejecutamos:

\begin{verbatim}
openssl x509 -req -days 365 -in server.csr \
-signkey server.key -out server.crt
\end{verbatim}
\item
  Servidor de base de datos.

  Para usar \textbf{\emph{web2py}} es imprescindible tener acceso a un
  servidor de base de datos. Podemos usar \emph{MySQL} o \emph{MariaDB}
  por ejemplo. Pero para empezar rápidamente vamos a tirar de
  \href{https://www.sqlite.org/version3.html}{SQLite}, un servidor fácil
  de instalar potente y versátil. Es importante usar la versión 3 que
  introduce grandes mejoras sobre el antiguo \emph{SQLite}

\begin{verbatim}
sudo apt install sqlite3
\end{verbatim}
\item
  Arrancamos el servidor:

  Deberíamos tener los ficheros generados en el paso anterior:
  \texttt{server.key}, \texttt{server.csr} y \texttt{server.crt}, en el
  directorio raiz de web2py. Podemos arrancar el servidor con los
  siguientes parámetros (recuerda activar el entorno virtual si no lo
  tienes activo):

\begin{verbatim}
python web2py.py -a 'admin_password' -c server.crt -k server.key \
-i 0.0.0.0 -p 8000
\end{verbatim}

  Y ya podemos acceder nuestro server \textbf{\emph{web2py}}, con
  nuestro
  \href{https://www.mozilla.org/en-US/firefox/developer/}{navegador
  favorito}, visitando la dirección \url{https://localhost:8000}
\end{enumerate}

Y ahora si que ya tenemos todo listo para empezar a usar
\textbf{\emph{web2py}}. Podemos crear nuestra primera aplicación.

\hypertarget{los-detalles-tenebrosos}{%
\subsubsection{Los detalles tenebrosos}\label{los-detalles-tenebrosos}}

Si tienes mucha prisa por aprender web2py puedes saltarte esta sección e
ir directamente a la sección
\protect\hyperlink{nuestra-primera-aplicaciuxf3n}{siguiente}

Si por el contrario quieres entender exactamente que hemos hecho para
poder arrancar el \textbf{\emph{web2py}} continuar leyendo puede ser el
primer paso.

\begin{description}
\item[¿Qué es un \emph{virtualenv}?]
Python nos permite definir \emph{virtualenv}. Un \emph{virtualenv} es un
entorno python aislado. Todos los \emph{virtualenvs} están aislados
entre si y mejor todavía son independientes del python del sistema. Esto
te permite tener multiples entornos de desarrollo (o producción) cada
uno con distintas versiones de python y diferentes librerias python
instaladas en cada uno de ellos, o quizás diferentes versiones de las
mismas librerias.
\item[¿Que es \emph{virtualenvwrapper}?]
Es un frontend para usar \emph{virtualenv}, la herramienta nativa de
python para gestionar \emph{virtualenvs}. Es completamente opcional,
aunque a mi me parece muy cómoda.
\item[¿Qué es todo eso de los certificados?]
\textbf{\emph{web2py}} viene preparado para usar \emph{https} (estas
siglas tienen varias interpretaciones: \emph{HTTP over TLS}, \emph{HTTP
over SSL} o \emph{HTTP Secure}). \emph{https} usa comunicaciones
cifradas entre tu navegador y el servidor web para garantizar dos cosas:
que estás accediendo al auténtico servidor y que nadie este
interceptando la comunicación entre navegador y servidor. En particular
\textbf{\emph{web2py}} exige que se use \emph{https} para conectarse a
las páginas de administración. Así que si no generas los certificados
podrás arrancar y conectar con \textbf{\emph{web2py}} pero no podrás
hacer demasiadas cosas.

Para usar \emph{https} hay que hacer varias cosas:

\begin{itemize}
\tightlist
\item
  Generar un CSR (Certificate Signing Request)
\item
  Obtener con ese CSR un certificado SSL de una autoridad certificadora
  (CA)
\item
  O alternativamente generar nosotros un certificado a partir del CSR
\end{itemize}

Lo que hemos hecho con los comandos \emph{openssl} ha sido:

\begin{itemize}
\tightlist
\item
  Generar un par de claves (privada y pública) para nuestro servidor
  (\texttt{server.key})
\item
  Generar con esa clave un CSR (el CSR lleva la información que le hemos
  metido de nuestro servidor y la clave pública)
\item
  Generar un certificado firmándolo nosotros mismos con esa misma clave
  como si fueramos la autoridad certificadora.
\end{itemize}

Esto nos vale para arrancar \textbf{\emph{web2py}} aunque nuestro
navegador nos dará una alerta de riesgo de seguridad por que no reconoce
a la CA.
\end{description}

\href{https://www.digitalocean.com/community/tutorials/openssl-essentials-working-with-ssl-certificates-private-keys-and-csrs}{Más
info de \emph{openssl}}

\begin{description}
\item[¿Qué es un motor de base de datos?]
\textbf{\emph{web2py}} usa un motor (o gestor) de
\href{https://es.wikipedia.org/wiki/Base_de_datos_relacional}{base de
datos relacional}. Puede usar muchos, incluyendo los más populares como
por ejemplo MySQL, Postgres o MariaDB.

Las bases de datos relacionales se basan en relaciones. Las relaciones
primarias son tablas que almacenan registros (filas) con atributos
comunes (columnas). Las relaciones derivadas se establecen entre
distintas tablas mediante consultas (queries) o vistas (views)

\textbf{\emph{web2py}} te permite gestionar y utilizar las bases de
datos a muy alto nivel, así que podras usarlo sin saber practicamente de
bases de datos; pero no es demasiado difícil aprender los conceptos
básicos y compensa ;-) Todo lo que puedas aprender de bases de datos te
ayudará a hacer mejores aplicaciones web.
\end{description}

\hypertarget{nuestra-primera-aplicaciuxf3n}{%
\subsection{Nuestra primera
aplicación}\label{nuestra-primera-aplicaciuxf3n}}

Vamos a crear nuestra primera aplicación en web2py.

Si has seguido los pasos de la
\protect\hyperlink{instalaciuxf3n}{sección anterior} ya tienes el
\textbf{\emph{web2py}} funcionando y puedes seguir cualquiera de los
tutoriales que hay en la red para aprender. El
\href{http://web2py.com/books/default/chapter/29/03/overview}{capítulo
3} del libro de \textbf{\emph{web2py}} es muy recomendable, y está
disponible
\href{http://web2py.com/books/default/chapter/41/03/resumen}{en
castellano}, puedes ventilarte los ejemplos que trae explicados en una
tarde y son muy ilustrativos.

En esta guía vamos a ver la creación de una aplicación paso a paso.
Crearemos una aplicación de inventario para el material de la Asociación
BricoLabs, pero lo haremos de manera que también nos valga para uso
particular y tener controladas todas nuestras cacharradas.

Este no es un tutorial de diseño profesional de aplicaciones, solo
pretendemos demostrar lo fácil que es iniciarse con
\textbf{\emph{web2py}}.

De hecho, no seguiremos un orden lógico en el diseño de la aplicación,
si no que intentaremos seguir un orden que facilite conocer el
framework.

Sin más rollo, vamos a comenzar con nuestra aplicación:

Crea una aplicación desde el interfaz de administración, en nuestro caso
la llamaremos \textbf{\emph{cornucopia}}.

Nuestro \textbf{\emph{web2py}} ``viene de serie'' con algunas
aplicaciones de ejemplo. La propia pantalla inicial es una de ellas la
aplicación ``Welcome'' o ``Bienvenido'' (dependerá del lenguaje por
defecto de tu navegador).

Para crear nuestra aplicación \textbf{\emph{cornucopia}}:

\begin{itemize}
\tightlist
\item
  Vamos al botón \textbf{admin} en la pantalla principal.
\item
  Metemos la password de administración (con la que hemos arrancado el
  \textbf{\emph{web2py}} en la linea de comandos).
\item
  Desde la ventana de administración creamos nuestra nueva aplicación
\end{itemize}

Inmediatamente nos encontraremos en la ventana de diseño de nuestra
nueva aplicación. \textbf{\emph{web2py}} nos permite diseñar
completamente nuestra aplicación desde aquí, ni siquiera necesitaremos
un editor de texto (aunque nada impide usar uno, desde luego).

\hypertarget{privateappconfig.ini}{%
\subsubsection{\texorpdfstring{\texttt{private/appconfig.ini}}{private/appconfig.ini}}\label{privateappconfig.ini}}

El primer fichero que vamos a examinar es \texttt{private/appconfig.ini}
La sección \texttt{private} debería estar abajo de todo en la ventana de
diseño.

En la sección \texttt{{[}app{]}} del fichero podemos configurar el
nombre de la aplicación y los datos del desarrollador.

En la sección \texttt{{[}db{]}} fichero configuramos el motor de base de
datos que vamos a usar en nuestra aplicación. Por defecto viene
configurado \emph{sqlite} así que no vamos a tener que cambiar nada en
este sentido.

En la seccion \texttt{{[}smtp{]}} podemos configurar el gateway de
correo que usará la aplicación para enviar correos a los usuarios. Por
defecto viene viene la configuración para usar una cuenta de gmail como
gateway, solo tenemos que cubrir los valores de usuario y password y la
dirección de correo.\footnote{Es aconsejable crear una cuenta de gmail,
  o cualquier otro servicio de correo que nos guste, para pruebas. Usar
  tu cuenta de correo personal podría ser muy mala idea}

\hypertarget{el-modelo}{%
\subsubsection{El Modelo}\label{el-modelo}}

En la parte superior de la ventana de diseño (o edición) de nuestra
aplicación tenemos la sección \texttt{Models}

\begin{figure}
\centering
\includegraphics{src/img/models_menu.jpg}
\caption{Menú Modelos}
\end{figure}

\textbf{\emph{web2py}} se encarga de crear las tablas necesarias en la
base de datos que le hayamos indicado que use.

Al crear la aplicación \_\textbf{web2py} ha creado en la base de datos
todas las tablas relacionadas con la gestión de usuarios y sus
privilegios.

Si echamos un ojo al modelo gráfico (\emph{Graphs Models}) veremos las
tablas que \textbf{\emph{web2py}} ha creado por defecto y las relaciones
entre ellas. Estas tablas que ha creado el \emph{framework} son las que
se encargan de la gestión de usuarios, sus privilegios y el acceso de
los mismos al sistema, es decir la capa de seguridad.

Si vemos el log de comandos de sql (\emph{sql.log}) veremos los comandos
que \textbf{\emph{web2py}} ha ejecutado en el motor de base de datos.

Y por último si vemos \emph{database administration} podremos ver las
tablas creadas en la base de datos, e incluso crear nuevos registros en
esas tablas (de momento no lo hagas)

También podemos echar un ojo al contenido del fichero \texttt{db.py} o
\texttt{menu.py} pero por el momento \textbf{no} vamos a modificar nada
en esos ficheros.

Ahora tenemos que ampliar el modelo y añadir todo lo que consideremos
necesario para nuestra aplicación.

\hypertarget{diseuxf1ando-el-modelo}{%
\paragraph{Diseñando el modelo}\label{diseuxf1ando-el-modelo}}

\emph{Build fat models and thin controllers} es uno de los lemas del
modelo MVC, no vamos a entrar en detalles de momento pero un modelo bien
diseñado nos va a ahorrar muchísimo trabajo al construir la aplicación.

El diseño de bases de datos es una rama de la ingeniería en si mismo,
hay camiones de libros escritos sobre el tema y todo tipo de
herramientas para ayudar al diseñador. Pero nosotros nos vamos a centrar
en usar sólo lo que nos ofrece \textbf{\emph{web2py}}.

Además como estamos aprendiendo vamos a ver algunas facilidades que nos
da \textbf{\emph{web2py}} sin proponer ningún proceso de diseño del
modelo (recuerda, esto no es un curso de diseño de aplicaciones)

Vamos a definir el modelo (concretamente las tablas) de nuestra
aplicación en un nuevo fichero de la sección \emph{Models}, que
llamaremos \texttt{db\_custom} así que pulsamos en el botón
\emph{Create}, y creamos el fichero \texttt{db\_custom}.

\begin{figure}
\centering
\includegraphics{src/img/create_db_custom.png}
\caption{Crear fichero}
\end{figure}

\textbf{\emph{web2py}} parsea todos los ficheros de la sección
\emph{Models} por orden alfabético. Esto nos permite separar nuestro
código del que viene originalmente con la aplicación. Pero es importante
que \texttt{db.py} sea siempre el primero alfabeticamente para que se
ejecute antes que el resto.

\textbf{\emph{web2py}} se encarga también de añadir la extensión
\texttt{.py} al nuevo fichero que estamos creando así que teclea sólo el
nombre \texttt{db\_custom}.

El objetivo de nuestra aplicación es mantener un inventario de
``cosas''. Parece lógico que nuestra primera tabla valga para almacenar
``cosas''. Así que en el fichero \texttt{db\_custom.py} añadimos las
siguientes lineas y salvamos el fichero:

\begin{verbatim}
db.define_table('thing',
    Field('id', 'integer'),
    Field('desc', 'string'),
    migrate = True);
\end{verbatim}

Ya hemos salvado nuestro fichero, vamos a echar un ojo a nuestra base de
datos con el botón \emph{Graph Model}.

\begin{figure}
\centering
\includegraphics{src/img/internal_error.jpg}
\caption{Error interno}
\end{figure}

¡Tenemos un horror! ¿Qué ha pasado?. Si pinchamos en el link del
\emph{Ticket} se abrirá una nueva pestaña en nuestro navegador:

\begin{figure}
\centering
\includegraphics{src/img/error_reserved_sql.jpg}
\caption{Error palabra reservada SQL}
\end{figure}

En el ticket tenemos mucha información acerca del error, afortunadamente
en este caso es facilito. El nombre del campo \texttt{desc} que hemos
añadido a nuestra tabla \texttt{thing} es una palabra reservada en
\textbf{todas} las variedades de \emph{SQL} (es el comando para ver la
definición de una tabla: \emph{desc tablename})

Editamos de nuevo nuestro fichero \texttt{db\_custom.py} y corregimos el
contenido:

\begin{verbatim}
db.define_table('thing',
    Field('id', 'integer'),
    Field('description', 'string'),
    migrate = True);
\end{verbatim}

¡Ahora si! Si pulsamos en el botón de \emph{Graph Model} (después de
salvar el nuevo contenido) veremos que \textbf{\emph{web2py}} ha creado
la nueva tabla en la base de datos. Incluso podríamos empezar a añadir
filas (cosas) a nuestra tabla desde el \emph{database administration}

El campo \texttt{id} es \emph{casi} obligatorio en todas las tablas que
definamos en \textbf{\emph{web2py}}, siempre será un valor único para
cada fila en una tabla y se usará internamente como clave primaria.
Podemos usar otros campos como clave primaria pero de momento
mantendremos las cosas símples.

Si visitas ahora la sección de administración de la base de datos puedes
añadir algunas ``cosas'' a la nueva tabla.

\begin{figure}
\centering
\includegraphics{src/img/add_thing_a.jpg}
\caption{Añadir destornillador}
\end{figure}

Evidentemente nuestro modelo de ``cosa'' es demasiado simple, tenemos
que añadirle nuevos atributos de
\href{http://web2py.com/books/default/chapter/29/06/the-database-abstraction-layer\#Field-types}{distintos
tipos} para que sea funcional. Pero antes de ir a por todas vamos a ver
algunas funciones que nos ofrece \textbf{\emph{web2py}} para construir
los Modelos.

Vamos a añadir algunos campos más de distintos tipos a nuestro modelo y
verlos con un poco de calma.

\begin{verbatim}
db.define_table('thing',
    Field('id', 'integer'),
    Field('name', 'string'),
    Field('description', 'string'),
    Field('picture',, 'upload'),
    Field('created_on, 'datetime'),
    migrate = True);
\end{verbatim}

Si ahora volvemos al administrador de base de datos podemos comprobar
que:

\begin{itemize}
\item
  No hemos perdido las ``cosas'' que añadimos antes,
  \textbf{\emph{web2py}} ha añadido las nuevas columnas pero ha
  conservado los valores de las antiguas.
\item
  Podemos editar las ``cosas'' que habíamos añadido sin mas que hacer
  click en el \texttt{id}
\item
  Si queremos editar (o añadir) una ``cosa'', \textbf{\emph{web2py}} nos
  ofrece un diálogo para subir la foto de nuestro objeto. Sabe que los
  atributos de tipo upload son fichero que subiremos al servidor.

  De la misma forma nos ofrece un menú inteligente para añadir el campo
  \texttt{datetime}
\end{itemize}

Este es el tipo de facilidades que ofrecen los \emph{frameworks} para
acelerar el trabajo de crear una aplicación.

Vamos a hacer un pelín más sofisticada nuestra tabla \texttt{thing}:

\begin{verbatim}
db.define_table('thing',
    Field('id', 'integer'),
    Field('name', 'string', requires = IS_NOT_EMPTY(error_message='cannot be empty')),
    Field('description', 'string'),
    Field('qty', 'integer', default=1, label=T('Quantity')),
    Field('picture', 'upload'),
    Field('created_on', 'datetime'),
    format='%(name)s',
    migrate = True);
\end{verbatim}

En la linea del \texttt{name} hemos añadido un \texttt{VALIDATOR}. Se
trata de
\href{http://web2py.com/books/default/chapter/29/07/forms-and-validators}{funciones
auxiliares} que nos permiten comprobar multitud de condiciones y que son
extremadamente útiles (iremos viendo casos de uso). En este caso
exigimos que el campo \texttt{name} no puede estar vacío y además
especificamos el mensaje de error que debe aparecer si sucede.

Hemos añadido un atributo \texttt{qty} (cantidad), hemos especificado
que tenga un valor por defecto de una unidad, y además hemos
especificado el \texttt{label}.

El \texttt{label} se usará en los formularios en lugar del nombre del
campo en la base de datos. Si vamos a añadir una nueva ``cosa'' veremos
que en el formulario no aparece \emph{qty} sino que nos pregunta
\emph{Quantity}. Además, y esto es muy importante, hemos asignado el
valor de la etiqueta con la función \texttt{T()}.

\textbf{\emph{web2py}} incorpora un sistema completo de
internacionalización. Al usar la función \texttt{T()} la cadena
\emph{Quantity} se ha añadido a todos los diccionarios de traducción (si
es que no estaba ya) y solo tenemos que añadir la traducción en el
diccionario correspondiente (p.ej. a \texttt{es.py}) para que funcione
la i18n. Una vez añadida si el idioma por defecto de nuestro navegador
es el castellano, en el formulario aparecerá ``Cantidad'' en lugar de
\emph{Quantity}.

Por último hemos añadido el \texttt{format} a la definición de la tabla,
\texttt{format} especifica que cuando nos refiramos a un objeto ``cosa''
se represente por defecto con su atributo \texttt{name}.

\begin{verbatim}
db.define_table('thing',
    Field('id', 'integer'),
    Field('description', 'string'),
    Field('picture', 'upload'),
    Field('created_on, 'datetime'),
    Field('created_by, 'reference auth_user', default = auth.user_id),
    Field('updated_on, 'datetime'),
    migrate = True);
\end{verbatim}

\hypertarget{secciones-en-el-futuro}{%
\section{Secciones en el futuro}\label{secciones-en-el-futuro}}

\hypertarget{web2py-y-git}{%
\subsection{web2py y git}\label{web2py-y-git}}

\hypertarget{instalaciuxf3n-con-nginx}{%
\subsection{Instalación con nginx}\label{instalaciuxf3n-con-nginx}}

\hypertarget{certificados-lets-encrypt}{%
\subsection{Certificados let's
encrypt}\label{certificados-lets-encrypt}}

\end{document}
